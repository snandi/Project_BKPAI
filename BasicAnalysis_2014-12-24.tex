%\documentclass[11pt]{extarticle} %extarticle for fontsizes other than 10, 11 And 12
\documentclass[11pt]{article}

%%%%%%%%%%%%%%%%%%%%%%%% Packages %%%%%%%%%%%%%%%%%%%%%%%%
\usepackage{amscd}
\usepackage{amsmath}
\usepackage{amssymb}
\usepackage{amsthm}
\usepackage{amsxtra}
\usepackage{bbold}
%\usepackage{bigints}
\usepackage{color}
\usepackage{dsfont}
\usepackage{enumerate}
\usepackage[mathscr]{eucal}
%\usepackage{fancyhdr}
\usepackage{float}
%\usepackage{fullpage} %% Dont use this for beamer presentations
\usepackage{geometry}
\usepackage{graphicx}
\usepackage{hyperref}
\usepackage{indentfirst}
\usepackage{latexsym}
\usepackage{listings}
\usepackage{lscape}
%\usepackage{mathtools}
\usepackage{microtype}
\usepackage{natbib}
\usepackage{pdfpages}
\usepackage{verbatim}
\usepackage{wrapfig}
\usepackage{xargs}
\DeclareGraphicsExtensions{.pdf,.png,.jpg, .jpeg}

%%%%%%%%%%%%%%%%%%%%%%%% Commands %%%%%%%%%%%%%%%%%%%%%%%%
\newcommand{\Sup}{\textsuperscript}
\newcommand{\Exp}{\mathds{E}}
\newcommand{\Prob}{\mathds{P}}
\newcommand{\Z}{\mathds{Z}}
\newcommand{\Ind}{\mathds{1}}
\newcommand{\A}{\mathcal{A}}
\newcommand{\F}{\mathcal{F}}
%\newcommand{\G}{\mathcal{G}}
\newcommand{\I}{\mathcal{I}}
\newcommand{\be}{\begin{equation}}
\newcommand{\ee}{\end{equation}}
\newcommand{\bes}{\begin{equation*}}
\newcommand{\ees}{\end{equation*}}
\newcommand{\union}{\bigcup}
\newcommand{\intersect}{\bigcap}
\newcommand{\Ybar}{\overline{Y}}
\newcommand{\ybar}{\bar{y}}
\newcommand{\Xbar}{\overline{X}}
\newcommand{\xbar}{\bar{x}}
\newcommand{\betahat}{\hat{\beta}}
\newcommand{\Yhat}{\widehat{Y}}
\newcommand{\yhat}{\hat{y}}
\newcommand{\Xhat}{\widehat{X}}
\newcommand{\xhat}{\hat{x}}
\newcommand{\E}[1]{\operatorname{E}\left[ #1 \right]}
%\newcommand{\Var}[1]{\operatorname{Var}\left( #1 \right)}
\newcommand{\Var}{\operatorname{Var}}
\newcommand{\Cov}[2]{\operatorname{Cov}\left( #1,#2 \right)}
\newcommand{\N}[2][1=\mu, 2=\sigma^2]{\operatorname{N}\left( #1,#2 \right)}
\newcommand{\bp}[1]{\left( #1 \right)}
\newcommand{\bsb}[1]{\left[ #1 \right]}
\newcommand{\bcb}[1]{\left\{ #1 \right\}}
\newcommand*{\permcomb}[4][0mu]{{{}^{#3}\mkern#1#2_{#4}}}
\newcommand*{\perm}[1][-3mu]{\permcomb[#1]{P}}
\newcommand*{\comb}[1][-1mu]{\permcomb[#1]{C}}

%%%%%%%%%%%%%%%%%%% To change the margins and stuff %%%%%%%%%%%%%%%%%%%
\geometry{left=1in, right=1in, top=1in, bottom=0.8in}
%\setlength{\voffset}{0.5in}
%\setlength{\hoffset}{-0.4in}
%\setlength{\textwidth}{7.6in}
%\setlength{\textheight}{10in}
%%%%%%%%%%%%%%%%%%%%%%%%%%%%%%%%%%%%%%%%%%%%%%%%%%%%%%%%%%%%%%%%%%%%%%%

\usepackage{Sweave}
\begin{document}
\input{BasicAnalysis_2014-12-24-concordance}

\bibliographystyle{plain}  %Choose a bibliograhpic style

\title{Analyzing the BKPAI Data}
\author{Subhrangshu Nandi\\
  Stat 760: Multivariate Analysis\\
  Project Report \\
  nandi@stat.wisc.edu}
\date{November 25, 2014}
%\date{}

%\maketitle

\begin{center}
{\Large{Analyzing the BKPAI Data}}\\
%Subhrangshu Nandi\\
%Stat 760: Multivariate Analysis\\
%Project Proposal\\
December 24, 2014
\end{center}

\subsection*{Wealth quintile}
\begin{center}
\includegraphics{BasicAnalysis_2014-12-24-002}
\end{center}
Interpretation of this figure is that $60\%$ people in wealth quintile 1, live with children, whereas about $85\%$ people in wealth quintiles 4 and 5 live with children.

\newpage
\subsection*{Poor Health}
Frequency of the Poor health variable:
\begin{Schunk}
\begin{Soutput}
  healthy adults unhealthy adults 
            3390             4516 
\end{Soutput}
\end{Schunk}
Now, lets check for independence between Poor health and living arrangement:
\begin{Schunk}
\begin{Soutput}
                   others alone spouse only with children
  healthy adults      103   250         460          2577
  unhealthy adults    201   313         516          3486
\end{Soutput}
\end{Schunk}

% latex table generated in R 3.1.1 by xtable 1.7-4 package
% Wed Dec 24 22:34:39 2014
\begin{table}[ht]
\centering
\begin{tabular}{rrrr}
  \hline
 & X\_Sq & DF & pvalue \\ 
  \hline
X-squared & 18.14 & 3 & 0.000 \\ 
   \hline
\end{tabular}
\end{table}Conclusion: $\chi^2$ test rejects the null hypothesis of independence between poor health and living arrangement. 

\newpage
\subsection*{Total Income}
\begin{center}
\includegraphics{BasicAnalysis_2014-12-24-006}
\end{center}
Since the variability in the $row\_total\_income$ is too high, the boxplot of the $\log$ of the variable is displayed. The living arrangement with ``spouse-only'' seems to have highest average income. There are lot of aging adults with no income in the first and fourth type of living arrangement.


\newpage
\subsection*{Depend Child ($depend\_child\_mx$)}
% latex table generated in R 3.1.1 by xtable 1.7-4 package
% Wed Dec 24 22:34:40 2014
\begin{table}[ht]
\centering
\begin{tabular}{rrr}
  \hline
 & no & yes \\ 
  \hline
1 & 3335 & 4585 \\ 
   \hline
\end{tabular}
\end{table}% latex table generated in R 3.1.1 by xtable 1.7-4 package
% Wed Dec 24 22:34:40 2014
\begin{table}[ht]
\centering
\begin{tabular}{rrrrr}
  \hline
 & others & alone & spouse only & with children \\ 
  \hline
no & 206 & 398 & 742 & 1989 \\ 
  yes &  98 & 167 & 234 & 4086 \\ 
   \hline
\end{tabular}
\end{table}Now, lets check for independence between depend child and living arrangement:
% latex table generated in R 3.1.1 by xtable 1.7-4 package
% Wed Dec 24 22:34:40 2014
\begin{table}[ht]
\centering
\begin{tabular}{rrrr}
  \hline
 & X\_Sq & DF & pvalue \\ 
  \hline
X-squared & 947.39 & 3 & 0.000 \\ 
   \hline
\end{tabular}
\end{table}
\begin{center}
\includegraphics{BasicAnalysis_2014-12-24-009}
\end{center}

\newpage
\subsection*{Own Asset}

% latex table generated in R 3.1.1 by xtable 1.7-4 package
% Wed Dec 24 22:34:40 2014
\begin{table}[ht]
\centering
\begin{tabular}{rrr}
  \hline
 & 0 & 1 \\ 
  \hline
1 & 1514 & 6406 \\ 
   \hline
\end{tabular}
\end{table}% latex table generated in R 3.1.1 by xtable 1.7-4 package
% Wed Dec 24 22:34:40 2014
\begin{table}[ht]
\centering
\begin{tabular}{rrr}
  \hline
 & 0 & 1 \\ 
  \hline
others &  70 & 234 \\ 
  alone & 115 & 450 \\ 
  spouse only & 117 & 859 \\ 
  with children & 1212 & 4863 \\ 
   \hline
\end{tabular}
\end{table}% latex table generated in R 3.1.1 by xtable 1.7-4 package
% Wed Dec 24 22:34:40 2014
\begin{table}[ht]
\centering
\begin{tabular}{rrrr}
  \hline
 & X\_Sq & DF & pvalue \\ 
  \hline
X-squared & 38.38 & 3 & 0.000 \\ 
   \hline
\end{tabular}
\end{table}Conclusion: $\chi^2$ test rejects the null hypothesis of independence between asset ownership and living arrangement. 

\newpage
\subsection*{religion}
% latex table generated in R 3.1.1 by xtable 1.7-4 package
% Wed Dec 24 22:34:40 2014
\begin{table}[ht]
\centering
\begin{tabular}{rrrrrr}
  \hline
 & hindu & muslim & christian & sikh & other \\ 
  \hline
1 & 6272 & 680 & 268 & 600 & 100 \\ 
   \hline
\end{tabular}
\end{table}% latex table generated in R 3.1.1 by xtable 1.7-4 package
% Wed Dec 24 22:34:40 2014
\begin{table}[ht]
\centering
\begin{tabular}{rrrrr}
  \hline
 & others & alone & spouse only & with children \\ 
  \hline
hindu & 0.03 & 0.08 & 0.13 & 0.76 \\ 
  muslim & 0.06 & 0.06 & 0.06 & 0.83 \\ 
  christian & 0.06 & 0.07 & 0.16 & 0.71 \\ 
  sikh & 0.05 & 0.03 & 0.11 & 0.81 \\ 
  other & 0.01 & 0.09 & 0.13 & 0.77 \\ 
   \hline
\end{tabular}
\end{table}% latex table generated in R 3.1.1 by xtable 1.7-4 package
% Wed Dec 24 22:34:40 2014
\begin{table}[ht]
\centering
\begin{tabular}{rrrr}
  \hline
 & X\_Sq & DF & pvalue \\ 
  \hline
X-squared & 70.39 & 12 & 0.0000 \\ 
   \hline
\end{tabular}
\end{table}Conclusion: $\chi^2$ test rejects the null hypothesis of independence between religion and living arrangement. 

\newpage
\subsection*{Disability ($disability\_mx$)}
% latex table generated in R 3.1.1 by xtable 1.7-4 package
% Wed Dec 24 22:34:40 2014
\begin{table}[ht]
\centering
\begin{tabular}{rrr}
  \hline
 & no & yes \\ 
  \hline
1 & 1905 & 6015 \\ 
   \hline
\end{tabular}
\end{table}% latex table generated in R 3.1.1 by xtable 1.7-4 package
% Wed Dec 24 22:34:40 2014
\begin{table}[ht]
\centering
\begin{tabular}{rrr}
  \hline
 & no & yes \\ 
  \hline
others &  59 & 245 \\ 
  alone & 174 & 391 \\ 
  spouse only & 316 & 660 \\ 
  with children & 1356 & 4719 \\ 
   \hline
\end{tabular}
\end{table}% latex table generated in R 3.1.1 by xtable 1.7-4 package
% Wed Dec 24 22:34:40 2014
\begin{table}[ht]
\centering
\begin{tabular}{rrrr}
  \hline
 & X\_Sq & DF & pvalue \\ 
  \hline
X-squared & 64.65 & 3 & 0.0000 \\ 
   \hline
\end{tabular}
\end{table}Conclusion: $\chi^2$ test rejects the null hypothesis of independence between disability and living arrangement. 

\newpage
\subsection*{Work Adult}
% latex table generated in R 3.1.1 by xtable 1.7-4 package
% Wed Dec 24 22:34:40 2014
\begin{table}[ht]
\centering
\begin{tabular}{rrr}
  \hline
 & no & yes \\ 
  \hline
1 & 5629 & 2291 \\ 
   \hline
\end{tabular}
\end{table}% latex table generated in R 3.1.1 by xtable 1.7-4 package
% Wed Dec 24 22:34:40 2014
\begin{table}[ht]
\centering
\begin{tabular}{rrr}
  \hline
 & no & yes \\ 
  \hline
others & 222 &  82 \\ 
  alone & 391 & 174 \\ 
  spouse only & 554 & 422 \\ 
  with children & 4462 & 1613 \\ 
   \hline
\end{tabular}
\end{table}% latex table generated in R 3.1.1 by xtable 1.7-4 package
% Wed Dec 24 22:34:40 2014
\begin{table}[ht]
\centering
\begin{tabular}{rrrr}
  \hline
 & X\_Sq & DF & pvalue \\ 
  \hline
X-squared & 115.42 & 3 & 0.0000 \\ 
   \hline
\end{tabular}
\end{table}Conclusion: $\chi^2$ test rejects the null hypothesis of independence between working adult and living arrangement. 

\newpage
\subsection*{Transfer child}
% latex table generated in R 3.1.1 by xtable 1.7-4 package
% Wed Dec 24 22:34:40 2014
\begin{table}[ht]
\centering
\begin{tabular}{rrr}
  \hline
 & no & yes \\ 
  \hline
1 & 7495 & 425 \\ 
   \hline
\end{tabular}
\end{table}% latex table generated in R 3.1.1 by xtable 1.7-4 package
% Wed Dec 24 22:34:40 2014
\begin{table}[ht]
\centering
\begin{tabular}{rrr}
  \hline
 & no & yes \\ 
  \hline
others & 286 &  18 \\ 
  alone & 529 &  36 \\ 
  spouse only & 927 &  49 \\ 
  with children & 5753 & 322 \\ 
   \hline
\end{tabular}
\end{table}% latex table generated in R 3.1.1 by xtable 1.7-4 package
% Wed Dec 24 22:34:40 2014
\begin{table}[ht]
\centering
\begin{tabular}{rrrr}
  \hline
 & X\_Sq & DF & pvalue \\ 
  \hline
X-squared & 1.59 & 3 & 0.6615 \\ 
   \hline
\end{tabular}
\end{table}Conclusion: $\chi^2$ test {\bf{fails to reject}} the null hypothesis of independence between ``transfer child'' and living arrangement. 

\newpage
\subsection*{Multinomial logit model}
A multinomial logit model was fit, with the following factors. Each of the factors were individually taken out of the model and a $\chi ^2$ test was conducted to evaulate its inclusion in the model. Below is the table of all the variables in the model, in descending order of their predictive power of the outcome variable (ascending order of their p values). 


% latex table generated in R 3.1.1 by xtable 1.7-4 package
% Wed Dec 24 22:34:46 2014
\begin{table}[ht]
\centering
\begin{tabular}{lr}
  \hline
Variable & pValue \\ 
  \hline
depend\_child\_max & 0.000000 \\ 
  poor\_health & 0.000000 \\ 
  religion & 0.000000 \\ 
  caste & 0.000000 \\ 
  work\_adult & 0.000001 \\ 
  own\_asset & 0.000147 \\ 
  disability\_mx & 0.000192 \\ 
  transfer\_child\_max & 0.306809 \\ 
   \hline
\end{tabular}
\end{table}
The conclusion is consistent with all the previous $\chi^2$ tests. Only ``transfer\_child\_max'' is not significant. 
\\
Comments:
\begin{enumerate}
\item This is just the first modeling attempt. I can include other variables if you are interested
\item This model does not account for interaction between the factors
\item There could be multicollinearity as well
\item In multinomial models, with many multi-level factors as independent variables, extracting and interpreting the coefficients is often very tricky. Once you decide which variables (factors) to use, send me a list to re-generate this result for. 
\end{enumerate}
\end{document}
