%\documentclass[11pt]{extarticle} %extarticle for fontsizes other than 10, 11 And 12
\documentclass[11pt]{article}

%%%%%%%%%%%%%%%%%%%%%%%% Packages %%%%%%%%%%%%%%%%%%%%%%%%
\usepackage{amscd}
\usepackage{amsmath}
\usepackage{amssymb}
\usepackage{amsthm}
\usepackage{amsxtra}
\usepackage{bbold}
%\usepackage{bigints}
\usepackage{color}
\usepackage{dsfont}
\usepackage{enumerate}
\usepackage[mathscr]{eucal}
%\usepackage{fancyhdr}
\usepackage{float}
%\usepackage{fullpage} %% Dont use this for beamer presentations
\usepackage{geometry}
\usepackage{graphicx}
\usepackage{hyperref}
\usepackage{indentfirst}
\usepackage{latexsym}
\usepackage{listings}
\usepackage{lscape}
%\usepackage{mathtools}
\usepackage{microtype}
\usepackage{natbib}
\usepackage{pdfpages}
\usepackage{verbatim}
\usepackage{wrapfig}
\usepackage{xargs}
\DeclareGraphicsExtensions{.pdf,.png,.jpg, .jpeg}

%%%%%%%%%%%%%%%%%%%%%%%% Commands %%%%%%%%%%%%%%%%%%%%%%%%
\newcommand{\Sup}{\textsuperscript}
\newcommand{\Exp}{\mathds{E}}
\newcommand{\Prob}{\mathds{P}}
\newcommand{\Z}{\mathds{Z}}
\newcommand{\Ind}{\mathds{1}}
\newcommand{\A}{\mathcal{A}}
\newcommand{\F}{\mathcal{F}}
%\newcommand{\G}{\mathcal{G}}
\newcommand{\I}{\mathcal{I}}
\newcommand{\be}{\begin{equation}}
\newcommand{\ee}{\end{equation}}
\newcommand{\bes}{\begin{equation*}}
\newcommand{\ees}{\end{equation*}}
\newcommand{\union}{\bigcup}
\newcommand{\intersect}{\bigcap}
\newcommand{\Ybar}{\overline{Y}}
\newcommand{\ybar}{\bar{y}}
\newcommand{\Xbar}{\overline{X}}
\newcommand{\xbar}{\bar{x}}
\newcommand{\betahat}{\hat{\beta}}
\newcommand{\Yhat}{\widehat{Y}}
\newcommand{\yhat}{\hat{y}}
\newcommand{\Xhat}{\widehat{X}}
\newcommand{\xhat}{\hat{x}}
\newcommand{\E}[1]{\operatorname{E}\left[ #1 \right]}
%\newcommand{\Var}[1]{\operatorname{Var}\left( #1 \right)}
\newcommand{\Var}{\operatorname{Var}}
\newcommand{\Cov}[2]{\operatorname{Cov}\left( #1,#2 \right)}
\newcommand{\N}[2][1=\mu, 2=\sigma^2]{\operatorname{N}\left( #1,#2 \right)}
\newcommand{\bp}[1]{\left( #1 \right)}
\newcommand{\bsb}[1]{\left[ #1 \right]}
\newcommand{\bcb}[1]{\left\{ #1 \right\}}
\newcommand*{\permcomb}[4][0mu]{{{}^{#3}\mkern#1#2_{#4}}}
\newcommand*{\perm}[1][-3mu]{\permcomb[#1]{P}}
\newcommand*{\comb}[1][-1mu]{\permcomb[#1]{C}}

%%%%%%%%%%%%%%%%%%% To change the margins and stuff %%%%%%%%%%%%%%%%%%%
\geometry{left=1in, right=1in, top=1in, bottom=0.8in}
%\setlength{\voffset}{0.5in}
%\setlength{\hoffset}{-0.4in}
%\setlength{\textwidth}{7.6in}
%\setlength{\textheight}{10in}
%%%%%%%%%%%%%%%%%%%%%%%%%%%%%%%%%%%%%%%%%%%%%%%%%%%%%%%%%%%%%%%%%%%%%%%

\usepackage{Sweave}
\begin{document}
\input{Report_2015_01-04_ContVar-concordance}

\bibliographystyle{plain}  %Choose a bibliograhpic style

\title{Analyzing the BKPAI Data}
\author{Subhrangshu Nandi\\
  Stat 760: Multivariate Analysis\\
  Project Report \\
  nandi@stat.wisc.edu}
\date{November 25, 2014}
%\date{}

%\maketitle

\begin{center}
{\Large{Analyzing the continuous independent variables of the BKPAI Data}}\\
%Subhrangshu Nandi\\
%Stat 760: Multivariate Analysis\\
%Project Proposal\\
January 4, 2015
\end{center}
Below is a table of the group means of ``Total income'', by different living arrangements:\\
% latex table generated in R 3.0.1 by xtable 1.7-1 package
% Sun Jan 04 11:40:25 2015
\begin{table}[ht]
\centering
\begin{tabular}{rrrrrrr}
  \hline
 & Min. & 1st Qu. & Median & Mean & 3rd Qu. & Max. \\ 
  \hline
others & 0 & 0 & 10000 & 142700.000 & 60000 & 9889000 \\ 
  alone & 0 & 1800 & 10000 & 109600.000 & 24000 & 9889000 \\ 
  spouse only & 0 & 9150 & 25000 & 165100.000 & 60000 & 9894000 \\ 
  with children & 0 & 0 & 3600 & 111800.000 & 42000 & 19330000 \\ 
   \hline
\end{tabular}
\end{table}\\
Below is a table of the group means of ``Average education'', by different living arrangements:\\
% latex table generated in R 3.0.1 by xtable 1.7-1 package
% Sun Jan 04 11:40:25 2015
\begin{table}[ht]
\centering
\begin{tabular}{rrrrrrrr}
  \hline
 & Min. & 1st Qu. & Median & Mean & 3rd Qu. & Max. & NA's \\ 
  \hline
others & 0 & 0 & 4 & 4.871 & 9 & 52 & 2 \\ 
  alone & 0 & 0 & 0 & 3.080 & 6 & 22 & 3 \\ 
  spouse only & 0 & 0 & 6 & 6.245 & 10 & 56 & 2 \\ 
  with children & 0 & 0 & 3 & 4.146 & 8 & 54 & 35 \\ 
   \hline
\end{tabular}
\end{table}\\
Below is a table of the group means of ``Total working adults'', by different living arrangements:\\

% latex table generated in R 3.0.1 by xtable 1.7-1 package
% Sun Jan 04 11:40:25 2015
\begin{table}[ht]
\centering
\begin{tabular}{rrrrrrr}
  \hline
 & Min. & 1st Qu. & Median & Mean & 3rd Qu. & Max. \\ 
  \hline
others & 0 & 0 & 0 & 0.572 & 1 & 4 \\ 
  alone & 0 & 0 & 0 & 0.000 & 0 & 0 \\ 
  spouse only & 0 & 0 & 0 & 0.075 & 0 & 1 \\ 
  with children & 0 & 0 & 1 & 1.037 & 1 & 9 \\ 
   \hline
\end{tabular}
\end{table}
\end{document}
